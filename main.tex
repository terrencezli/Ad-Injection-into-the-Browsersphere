
%% bare_jrnl.tex
%% V1.3
%% 2007/01/11
%% by Michael Shell
%% see http://www.michaelshell.org/
%% for current contact information.
%%
%% This is a skeleton file demonstrating the use of IEEEtran.cls
%% (requires IEEEtran.cls version 1.7 or later) with an IEEE journal paper.
%%
%% Support sites:
%% http://www.michaelshell.org/tex/ieeetran/
%% http://www.ctan.org/tex-archive/macros/latex/contrib/IEEEtran/
%% and
%% http://www.ieee.org/



% *** Authors should verify (and, if needed, correct) their LaTeX system  ***
% *** with the testflow diagnostic prior to trusting their LaTeX platform ***
% *** with production work. IEEE's font choices can trigger bugs that do  ***
% *** not appear when using other class files.                            ***
% The testflow support page is at:
% http://www.michaelshell.org/tex/testflow/


%%*************************************************************************
%% Legal Notice:
%% This code is offered as-is without any warranty either expressed or
%% implied; without even the implied warranty of MERCHANTABILITY or
%% FITNESS FOR A PARTICULAR PURPOSE! 
%% User assumes all risk.
%% In no event shall IEEE or any contributor to this code be liable for
%% any damages or losses, including, but not limited to, incidental,
%% consequential, or any other damages, resulting from the use or misuse
%% of any information contained here.
%%
%% All comments are the opinions of their respective authors and are not
%% necessarily endorsed by the IEEE.
%%
%% This work is distributed under the LaTeX Project Public License (LPPL)
%% ( http://www.latex-project.org/ ) version 1.3, and may be freely used,
%% distributed and modified. A copy of the LPPL, version 1.3, is included
%% in the base LaTeX documentation of all distributions of LaTeX released
%% 2003/12/01 or later.
%% Retain all contribution notices and credits.
%% ** Modified files should be clearly indicated as such, including  **
%% ** renaming them and changing author support contact information. **
%%
%% File list of work: IEEEtran.cls, IEEEtran_HOWTO.pdf, bare_adv.tex,
%%                    bare_conf.tex, bare_jrnl.tex, bare_jrnl_compsoc.tex
%%*************************************************************************

% Note that the a4paper option is mainly intended so that authors in
% countries using A4 can easily print to A4 and see how their papers will
% look in print - the typesetting of the document will not typically be
% affected with changes in paper size (but the bottom and side margins will).
% Use the testflow package mentioned above to verify correct handling of
% both paper sizes by the user's LaTeX system.
%
% Also note that the "draftcls" or "draftclsnofoot", not "draft", option
% should be used if it is desired that the figures are to be displayed in
% draft mode.
%
\documentclass[journal]{IEEEtran}
\usepackage{filecontents}
\usepackage{graphicx}

% Some very useful LaTeX packages include:
% (uncomment the ones you want to load)


% *** MISC UTILITY PACKAGES ***
%
%\usepackage{ifpdf}
% Heiko Oberdiek's ifpdf.sty is very useful if you need conditional
% compilation based on whether the output is pdf or dvi.
% usage:
% \ifpdf
%   % pdf code
% \else
%   % dvi code
% \fi
% The latest version of ifpdf.sty can be obtained from:
% http://www.ctan.org/tex-archive/macros/latex/contrib/oberdiek/
% Also, note that IEEEtran.cls V1.7 and later provides a builtin
% \ifCLASSINFOpdf conditional that works the same way.
% When switching from latex to pdflatex and vice-versa, the compiler may
% have to be run twice to clear warning/error messages.






% *** CITATION PACKAGES ***
%
%\usepackage{cite}
% cite.sty was written by Donald Arseneau
% V1.6 and later of IEEEtran pre-defines the format of the cite.sty package
% \cite{} output to follow that of IEEE. Loading the cite package will
% result in citation numbers being automatically sorted and properly
% "compressed/ranged". e.g., [1], [9], [2], [7], [5], [6] without using
% cite.sty will become [1], [2], [5]--[7], [9] using cite.sty. cite.sty's
% \cite will automatically add leading space, if needed. Use cite.sty's
% noadjust option (cite.sty V3.8 and later) if you want to turn this off.
% cite.sty is already installed on most LaTeX systems. Be sure and use
% version 4.0 (2003-05-27) and later if using hyperref.sty. cite.sty does
% not currently provide for hyperlinked citations.
% The latest version can be obtained at:
% http://www.ctan.org/tex-archive/macros/latex/contrib/cite/
% The documentation is contained in the cite.sty file itself.






% *** GRAPHICS RELATED PACKAGES ***
%
\ifCLASSINFOpdf
  % \usepackage[pdftex]{graphicx}
  % declare the path(s) where your graphic files are
  % \graphicspath{{../pdf/}{../jpeg/}}
  % and their extensions so you won't have to specify these with
  % every instance of \includegraphics
  % \DeclareGraphicsExtensions{.pdf,.jpeg,.png}
\else
  % or other class option (dvipsone, dvipdf, if not using dvips). graphicx
  % will default to the driver specified in the system graphics.cfg if no
  % driver is specified.
  % \usepackage[dvips]{graphicx}
  % declare the path(s) where your graphic files are
  % \graphicspath{{../eps/}}
  % and their extensions so you won't have to specify these with
  % every instance of \includegraphics
  % \DeclareGraphicsExtensions{.eps}
\fi
% graphicx was written by David Carlisle and Sebastian Rahtz. It is
% required if you want graphics, photos, etc. graphicx.sty is already
% installed on most LaTeX systems. The latest version and documentation can
% be obtained at: 
% http://www.ctan.org/tex-archive/macros/latex/required/graphics/
% Another good source of documentation is "Using Imported Graphics in
% LaTeX2e" by Keith Reckdahl which can be found as epslatex.ps or
% epslatex.pdf at: http://www.ctan.org/tex-archive/info/
%
% latex, and pdflatex in dvi mode, support graphics in encapsulated
% postscript (.eps) format. pdflatex in pdf mode supports graphics
% in .pdf, .jpeg, .png and .mps (metapost) formats. Users should ensure
% that all non-photo figures use a vector format (.eps, .pdf, .mps) and
% not a bitmapped formats (.jpeg, .png). IEEE frowns on bitmapped formats
% which can result in "jaggedy"/blurry rendering of lines and letters as
% well as large increases in file sizes.
%
% You can find documentation about the pdfTeX application at:
% http://www.tug.org/applications/pdftex





% *** MATH PACKAGES ***
%
%\usepackage[cmex10]{amsmath}
% A popular package from the American Mathematical Society that provides
% many useful and powerful commands for dealing with mathematics. If using
% it, be sure to load this package with the cmex10 option to ensure that
% only type 1 fonts will utilized at all point sizes. Without this option,
% it is possible that some math symbols, particularly those within
% footnotes, will be rendered in bitmap form which will result in a
% document that can not be IEEE Xplore compliant!
%
% Also, note that the amsmath package sets \interdisplaylinepenalty to 10000
% thus preventing page breaks from occurring within multiline equations. Use:
%\interdisplaylinepenalty=2500
% after loading amsmath to restore such page breaks as IEEEtran.cls normally
% does. amsmath.sty is already installed on most LaTeX systems. The latest
% version and documentation can be obtained at:
% http://www.ctan.org/tex-archive/macros/latex/required/amslatex/math/





% *** SPECIALIZED LIST PACKAGES ***
%
%\usepackage{algorithmic}
% algorithmic.sty was written by Peter Williams and Rogerio Brito.
% This package provides an algorithmic environment fo describing algorithms.
% You can use the algorithmic environment in-text or within a figure
% environment to provide for a floating algorithm. Do NOT use the algorithm
% floating environment provided by algorithm.sty (by the same authors) or
% algorithm2e.sty (by Christophe Fiorio) as IEEE does not use dedicated
% algorithm float types and packages that provide these will not provide
% correct IEEE style captions. The latest version and documentation of
% algorithmic.sty can be obtained at:
% http://www.ctan.org/tex-archive/macros/latex/contrib/algorithms/
% There is also a support site at:
% http://algorithms.berlios.de/index.html
% Also of interest may be the (relatively newer and more customizable)
% algorithmicx.sty package by Szasz Janos:
% http://www.ctan.org/tex-archive/macros/latex/contrib/algorithmicx/




% *** ALIGNMENT PACKAGES ***
%
%\usepackage{array}
% Frank Mittelbach's and David Carlisle's array.sty patches and improves
% the standard LaTeX2e array and tabular environments to provide better
% appearance and additional user controls. As the default LaTeX2e table
% generation code is lacking to the point of almost being broken with
% respect to the quality of the end results, all users are strongly
% advised to use an enhanced (at the very least that provided by array.sty)
% set of table tools. array.sty is already installed on most systems. The
% latest version and documentation can be obtained at:
% http://www.ctan.org/tex-archive/macros/latex/required/tools/


%\usepackage{mdwmath}
%\usepackage{mdwtab}
% Also highly recommended is Mark Wooding's extremely powerful MDW tools,
% especially mdwmath.sty and mdwtab.sty which are used to format equations
% and tables, respectively. The MDWtools set is already installed on most
% LaTeX systems. The lastest version and documentation is available at:
% http://www.ctan.org/tex-archive/macros/latex/contrib/mdwtools/


% IEEEtran contains the IEEEeqnarray family of commands that can be used to
% generate multiline equations as well as matrices, tables, etc., of high
% quality.


%\usepackage{eqparbox}
% Also of notable interest is Scott Pakin's eqparbox package for creating
% (automatically sized) equal width boxes - aka "natural width parboxes".
% Available at:
% http://www.ctan.org/tex-archive/macros/latex/contrib/eqparbox/





% *** SUBFIGURE PACKAGES ***
%\usepackage[tight,footnotesize]{subfigure}
% subfigure.sty was written by Steven Douglas Cochran. This package makes it
% easy to put subfigures in your figures. e.g., "Figure 1a and 1b". For IEEE
% work, it is a good idea to load it with the tight package option to reduce
% the amount of white space around the subfigures. subfigure.sty is already
% installed on most LaTeX systems. The latest version and documentation can
% be obtained at:
% http://www.ctan.org/tex-archive/obsolete/macros/latex/contrib/subfigure/
% subfigure.sty has been superceeded by subfig.sty.



%\usepackage[caption=false]{caption}
%\usepackage[font=footnotesize]{subfig}
% subfig.sty, also written by Steven Douglas Cochran, is the modern
% replacement for subfigure.sty. However, subfig.sty requires and
% automatically loads Axel Sommerfeldt's caption.sty which will override
% IEEEtran.cls handling of captions and this will result in nonIEEE style
% figure/table captions. To prevent this problem, be sure and preload
% caption.sty with its "caption=false" package option. This is will preserve
% IEEEtran.cls handing of captions. Version 1.3 (2005/06/28) and later 
% (recommended due to many improvements over 1.2) of subfig.sty supports
% the caption=false option directly:
%\usepackage[caption=false,font=footnotesize]{subfig}
%
% The latest version and documentation can be obtained at:
% http://www.ctan.org/tex-archive/macros/latex/contrib/subfig/
% The latest version and documentation of caption.sty can be obtained at:
% http://www.ctan.org/tex-archive/macros/latex/contrib/caption/




% *** FLOAT PACKAGES ***
%
%\usepackage{fixltx2e}
% fixltx2e, the successor to the earlier fix2col.sty, was written by
% Frank Mittelbach and David Carlisle. This package corrects a few problems
% in the LaTeX2e kernel, the most notable of which is that in current
% LaTeX2e releases, the ordering of single and double column floats is not
% guaranteed to be preserved. Thus, an unpatched LaTeX2e can allow a
% single column figure to be placed prior to an earlier double column
% figure. The latest version and documentation can be found at:
% http://www.ctan.org/tex-archive/macros/latex/base/



%\usepackage{stfloats}
% stfloats.sty was written by Sigitas Tolusis. This package gives LaTeX2e
% the ability to do double column floats at the bottom of the page as well
% as the top. (e.g., "\begin{figure*}[!b]" is not normally possible in
% LaTeX2e). It also provides a command:
%\fnbelowfloat
% to enable the placement of footnotes below bottom floats (the standard
% LaTeX2e kernel puts them above bottom floats). This is an invasive package
% which rewrites many portions of the LaTeX2e float routines. It may not work
% with other packages that modify the LaTeX2e float routines. The latest
% version and documentation can be obtained at:
% http://www.ctan.org/tex-archive/macros/latex/contrib/sttools/
% Documentation is contained in the stfloats.sty comments as well as in the
% presfull.pdf file. Do not use the stfloats baselinefloat ability as IEEE
% does not allow \baselineskip to stretch. Authors submitting work to the
% IEEE should note that IEEE rarely uses double column equations and
% that authors should try to avoid such use. Do not be tempted to use the
% cuted.sty or midfloat.sty packages (also by Sigitas Tolusis) as IEEE does
% not format its papers in such ways.


%\ifCLASSOPTIONcaptionsoff
%  \usepackage[nomarkers]{endfloat}
% \let\MYoriglatexcaption\caption
% \renewcommand{\caption}[2][\relax]{\MYoriglatexcaption[#2]{#2}}
%\fi
% endfloat.sty was written by James Darrell McCauley and Jeff Goldberg.
% This package may be useful when used in conjunction with IEEEtran.cls'
% captionsoff option. Some IEEE journals/societies require that submissions
% have lists of figures/tables at the end of the paper and that
% figures/tables without any captions are placed on a page by themselves at
% the end of the document. If needed, the draftcls IEEEtran class option or
% \CLASSINPUTbaselinestretch interface can be used to increase the line
% spacing as well. Be sure and use the nomarkers option of endfloat to
% prevent endfloat from "marking" where the figures would have been placed
% in the text. The two hack lines of code above are a slight modification of
% that suggested by in the endfloat docs (section 8.3.1) to ensure that
% the full captions always appear in the list of figures/tables - even if
% the user used the short optional argument of \caption[]{}.
% IEEE papers do not typically make use of \caption[]'s optional argument,
% so this should not be an issue. A similar trick can be used to disable
% captions of packages such as subfig.sty that lack options to turn off
% the subcaptions:
% For subfig.sty:
% \let\MYorigsubfloat\subfloat
% \renewcommand{\subfloat}[2][\relax]{\MYorigsubfloat[]{#2}}
% For subfigure.sty:
% \let\MYorigsubfigure\subfigure
% \renewcommand{\subfigure}[2][\relax]{\MYorigsubfigure[]{#2}}
% However, the above trick will not work if both optional arguments of
% the \subfloat/subfig command are used. Furthermore, there needs to be a
% description of each subfigure *somewhere* and endfloat does not add
% subfigure captions to its list of figures. Thus, the best approach is to
% avoid the use of subfigure captions (many IEEE journals avoid them anyway)
% and instead reference/explain all the subfigures within the main caption.
% The latest version of endfloat.sty and its documentation can obtained at:
% http://www.ctan.org/tex-archive/macros/latex/contrib/endfloat/
%
% The IEEEtran \ifCLASSOPTIONcaptionsoff conditional can also be used
% later in the document, say, to conditionally put the References on a 
% page by themselves.





% *** PDF, URL AND HYPERLINK PACKAGES ***
%
%\usepackage{url}
% url.sty was written by Donald Arseneau. It provides better support for
% handling and breaking URLs. url.sty is already installed on most LaTeX
% systems. The latest version can be obtained at:
% http://www.ctan.org/tex-archive/macros/latex/contrib/misc/
% Read the url.sty source comments for usage information. Basically,
% \url{my_url_here}.





% *** Do not adjust lengths that control margins, column widths, etc. ***
% *** Do not use packages that alter fonts (such as pslatex).         ***
% There should be no need to do such things with IEEEtran.cls V1.6 and later.
% (Unless specifically asked to do so by the journal or conference you plan
% to submit to, of course. )


% correct bad hyphenation here
\hyphenation{op-tical net-works semi-conduc-tor}


\begin{document}
%
% paper title
% can use linebreaks \\ within to get better formatting as desired
\title{Ad Injection into the Browsersphere}
%
%
% author names and IEEE memberships
% note positions of commas and nonbreaking spaces ( ~ ) LaTeX will not break
% a structure at a ~ so this keeps an author's name from being broken across
% two lines.
% use \thanks{} to gain access to the first footnote area
% a separate \thanks must be used for each paragraph as LaTeX2e's \thanks
% was not built to handle multiple paragraphs
%

\author{Terrence~Li~\IEEEmembership{California Polytechnic State University-San Luis Obispo}}

% note the % following the last \IEEEmembership and also \thanks - 
% these prevent an unwanted space from occurring between the last author name
% and the end of the author line. i.e., if you had this:
% 
% \author{....lastname \thanks{...} \thanks{...} }
%                     ^------------^------------^----Do not want these spaces!
%
% a space would be appended to the last name and could cause every name on that
% line to be shifted left slightly. This is one of those "LaTeX things". For
% instance, "\textbf{A} \textbf{B}" will typeset as "A B" not "AB". To get
% "AB" then you have to do: "\textbf{A}\textbf{B}"
% \thanks is no different in this regard, so shield the last } of each \thanks
% that ends a line with a % and do not let a space in before the next \thanks.
% Spaces after \IEEEmembership other than the last one are OK (and needed) as
% you are supposed to have spaces between the names. For what it is worth,
% this is a minor point as most people would not even notice if the said evil
% space somehow managed to creep in.



% The paper headers
% The only time the second header will appear is for the odd numbered pages
% after the title page when using the twoside option.
% 
% *** Note that you probably will NOT want to include the author's ***
% *** name in the headers of peer review papers.                   ***
% You can use \ifCLASSOPTIONpeerreview for conditional compilation here if
% you desire.




% If you want to put a publisher's ID mark on the page you can do it like
% this:
%\IEEEpubid{0000--0000/00\$00.00~\copyright~2007 IEEE}
% Remember, if you use this you must call \IEEEpubidadjcol in the second
% column for its text to clear the IEEEpubid mark.



% use for special paper notices
%\IEEEspecialpapernotice{(Invited Paper)}




% make the title area
\maketitle


\begin{abstract}
With cloud, we are beginning to see an increase in the reliance of browsers. So much so, the paper \textit{Ad Injection at Scale: Assessing Deceptive Advertisement Modifications} says that ``browsers are now analogous in function to operating systems'' \cite{ad_inject}. With the more dependence on the browser, we face new security threats regarding attacks on the browser. Such attacks are categorized as ``web injections''. This paper focuses on one of the subset of web injections called ``ad injections''. This paper will address the hack from a notorious advertisement company \textit{Superfish} in Lenovo laptops. This paper will address the social concerns, economic incentive, and technological impact of the Superfish hack. Additionally, the paper will address the technological failures that led to this hack in hopes to present mitigation techniques.


\end{abstract}
% IEEEtran.cls defaults to using nonbold math in the Abstract.
% This preserves the distinction between vectors and scalars. However,
% if the journal you are submitting to favors bold math in the abstract,
% then you can use LaTeX's standard command \boldmath at the very start
% of the abstract to achieve this. Many IEEE journals frown on math
% in the abstract anyway.

% Note that keywords are not normally used for peerreview papers.






% For peer review papers, you can put extra information on the cover
% page as needed:
% \ifCLASSOPTIONpeerreview
% \begin{center} \bfseries EDICS Category: 3-BBND \end{center}
% \fi
%
% For peerreview papers, this IEEEtran command inserts a page break and
% creates the second title. It will be ignored for other modes.
\IEEEpeerreviewmaketitle



\section{Introduction}
In 2015, Lenovo, a top-selling laptop brand, was caught pre-installing the software \textit{Superfish}. The controversy regarding the pre-installation of Superfish is that the software is described as malware or adware. Additionally, the US Department of Homeland Security recommended uninstalling this software and its root certificate because the software creates vulnerability to cyberattacks and interception of sensitive data \cite{superfish_wiki}.  

\subsection{The Rise of Ad Software}
Web injections occur when unwanted or malicious players modify browser sessions for personal gain. A subset of web injection are ad injections, which ads are forced onto users apart from the original website providers. Ad injectors are triggered through binary, extension, or network ISP, and modify a page's content to display ads without user consent \cite{ad_inject}. The prevalence of ad injections could be attributed to the fact this method is one of the most profitable ways to monetize browser traffic. We've seen evidences of injected ads in public WiFi via HTTP\cite{att_inject} or contained in the Yontoo browser plugin, earning them \$8 million\cite{ad_inject}. With the rise of these ad companies looking to capitalize on the potential for revenue, we've began to see companies, like Lenovo, looking for opportunities.

\section{Technical Explanation}
I will begin by defining and characterizing ad injection. This section will also discuss the social and technical failures of Lenovo regarding this hack.

\subsection{Ad Injection}
In this paper, I have borrowed Kurt Thomas' definition of ad injections as ``any binary, extension, or network ISP that modifies a page's content to insert or replace advertisements, irrespective of user consent''. The effects of ad injections are immediately apparent, as users are flooded with banner ads, scare campaign, and other intrusive behavior. For this reason, ad injectors have a negative impact on user experience, security, and privacy \cite{ad_inject}. 

User's browser experience is often affected because ad injectors increase page load latency by injecting third-party scripts that trigger XHR requests. These third-party scripts' purpose are meant to generate ad content from various ad network databases. Once these ads are fetched, these ads affect the page quality by flooding it with various banner ads, keyword highlights, and many other ``search results'' \cite{ad_inject}. Many of these ads are not from the original content provider and could appear out of place and overwhelm the users.

Besides the potential for malware, ad injectors pose additional threat to users by exposing their private data, such as browser history and page interaction, to other third-party applications.

These security concerns help segue the conversation to privacy concerns. Regarding privacy, many users are concerned about the fact that ad injectors often times monitor all of a user's browser activities. A user's browser activity includes their search history and page interaction. These data are then packaged by these third party ad injectors and either used to present advertisements or sold to their customers. These activities are usually done without the consent of the users and lies outside of the user's respective control.

\subsection{Lenovo Incident}
In 2015, Lenovo was discovered preloading the software Superfish. Superfish, an advertising company that was based in Palo Alto, was an ``adware'' that would inject it's own search results into a user's browsers. In addition to the injection of ads, the installed Superfish would also install a self-signed root certificate in the local trusted CA store \cite{lenovo_superfish}. 

According to Lenovo's official statement, Superfish was pre-installed with their laptops in an ``effort to provide a great user experience for our customers... The goal was to improve the shopping experience using their visual discovery techniques.''\cite{lenovo_statement}. Not only did Lenovo misunderstand their user base, the biggest social failure was that Lenovo used Superfish as a source of revenue. Looking to profit off these bloatware, Lenovo social mistake was thinking that Superfish was simply looking to present ads.

The technological failure is what alarmed users and security researchers. By providing Superfish a single self-signed root certificate, Superfish has the ability to intercept HTTP(S) traffic. Security analyst have demonstrated that this leaves laptops vulnerable to the man-in-the-middle attacks when using SSL. Essentially, when a user visits a site, Superfish can sign the site certificate and intercept the communication between the user and the website\cite{arstechnica_lenovo}. To make matters worse, it appears that the private key for Superfish-signed TLS certificate is the same for every Lenovo machine. The impacts of these failures are highlighted in this next section.

\section{Impact of the Superfish Hack}
This section will discuss the ramifications of the Superfish scandal in terms of social impact, economic impact, and technological impact. This scandal was a PR nightmare for Lenovo and affected their sales of laptop. Shortly after, Lenovo reached out immediately for a press release.

\subsection{Social Impact}
As mentioned, this incident was a PR nightmare for Lenovo. David Auerbach, a security researcher and writer, labeled this as one of the biggest scandals since the Sony DRM rootkit scandal of 2005 \cite{slate_lenovo}. Lenovo should have known about this problem since at least January 21, when a user posted a description of Superfish's exploit to the Lenovo forum. The question went unanswered for a month. Lenovo's CTO even proceeded to publicly claim that ``We’re not trying to get into an argument with the security guys. They’re dealing with theoretical concerns. We have no insight that anything nefarious has occurred.''\cite{slate_lenovo}. Notorious for its involvement in adware, spyware, and malware, Superfish denied any malpractices blaming ``false and misleading statements made by some media commentators and bloggers'' for tarnishing Superfish's image \cite{slate_lenovo}.

Besides the damage to Lenovo's reputation, there have been an individual lawsuit by Jesse Bennett. She charges Superfish and Lenovo with ``violating wiretap laws and trespassing on personal property''\cite{cnet_lenovo_lawsuit}. Bennett claims that Superfish tracked her browser activities and that violated her privacy and damaged her laptop. There has also been a class action investigation by Rosen Law Firm looking for affected Lenovo consumers. After the fallout, Lenovo has released an official statement stating that ``the principle that customer experience, security and privacy must be our top priorities. With this in mind, we will significantly reduce preloaded applications. Our goal is clear: to become the leader in providing cleaner, safer PCs.''

\subsection{Economic Impact}
While Lenovo claims to have installed Superfish to help enhance the user's shopping experience, it is clear that the main motivation was monetary gain. According to sources, the deal between Lenovo and Super fish was worth between \$200,000 and \$250,000 \cite{forbes_lenovo}. Judging by the fallout to Lenovo's reputation and legal ramifications, this deal seemed quite meager. It is hard to correlate Lenovo's stock prices with the Superfish incident because the stock price was already in steep decline due to smaller PC manufacturers and the need to install bloatware to be profitable \cite{dunn_2017}.

Few months after the incident, Superfish shut down its operations. The shutdown seemed to be more of a rebranding strategy as the Adi Pinahs, a co-founder of Superfish, started a new company called \textit{JustVisual}.

\subsection{Technological Impact}
Security researchers were able to discredit the claims made by Lenovo's CTO about the theoretical nature of Lenovo's security regarding Superfish. Robert Graham discussed the practical application of the security exploit by cracking the private key password. This challenge to Lenovo's statement demonstrated that anyone could use a private key to launch a man-in-the-middle HTTPS attacked that machines with the certificate can't detect \cite{arstechnica_lenovo}. Graham demonstrated that the certificate tampered with the HTTPS connection, which in his case was a banking site. This proved that the man-in-the-middle attack could be carried out practically by Superfish, allowing Superfish to collect unencrypted private data. Chris Palmer, a security researcher, also demonstrated that the characteristics of Superfish and its self-signed key could be manned in the middle to sign and issue a certificate to Bank of America's website, instead of the trusted CA VeriSign. Palmer confirmed that the private key for the Superfish certificate was the same key as another individual's Lenovo PC, meaning that attackers could use the vulnerable Lenovo machine's certificate to create fake HTTPS website. The technological impacts prove that there is a major security and privacy concern when HTTPS traffic could be intercepted and viewed unencrypted.

\section{Mitigation}
There has been recommendation for certificate pinning, but researchers like Rob Graham have demonstrated that this is not a defense against this attack. In fact, users who uninstall Superfish are still susceptible to the attack because it does not remove the root certificate \cite{slate_lenovo}. Lenovo have released updates that make sure the software will not be preloaded in the future and completely remove Superfish from their PCs. Additionally, Lenovo contacted Superfish to disable server side interaction on Lenovo products\cite{lenovo_superfish}.

Microsoft has also stepped in to change default Windows security software to detect and remove Superfish software\cite{cnet_lenovo}. CNET has also provided a Superfish removal guide. Since Lenovo had rated the Superfish vulnerability as ``high'', it's highest rating ever, it is highly recommended to follow the advisory listed on its support page\cite{lenovo_superfish}.

\section{Related Attacks}
Ad injections come in many forms. There have been ad injection incidents that lead to malware such as the scare campaigns that are injected into many sites. Google, in one of their analysis, has flagged 50,870 Chrome extensions as unwanted ad injectors, 38\% being malware. These attacks are extremely difficult to prevent as ad companies and many individuals are looking to profit from user's browser traffic. Consequently, ad injectors often market user browser data and personal information to advertisers\cite{ad_inject}. These attacks are likely to continue occurring, however there have been steps in reducing ads seen with tools such as Ad-Blocker, Ublock, etc.

\section{Conclusion}
In this paper, I have examined the properties of ad injectors and the specific Superfish incident with Lenovo. I have assessed the impacts of ad injectors on user's browser experience, security concerns, and privacy issues. Ad injectors look to monetize and capitalize on user browser traffic by injecting ads through various channels. Although most ad injections are just unwanted programs, some ad injections contain malware. 

Regarding the specific ad injection library, Superfish, I have discussed the impact of the Superfish and Lenovo scandal. Although Lenovo was looking to profit from the partnership deal with Superfish, Lenovo made a mistake in 




% if have a single appendix:
%\appendix[Proof of the Zonklar Equations]
% or
%\appendix  % for no appendix heading
% do not use \section anymore after \appendix, only \section*
% is possibly needed

% use appendices with more than one appendix
% then use \section to start each appendix
% you must declare a \section before using any
% \subsection or using \label (\appendices by itself
% starts a section numbered zero.)
%



% Can use something like this to put references on a page
% by themselves when using endfloat and the captionsoff option.
\ifCLASSOPTIONcaptionsoff
  \newpage
\fi



% trigger a \newpage just before the given reference
% number - used to balance the columns on the last page
% adjust value as needed - may need to be readjusted if
% the document is modified later
%\IEEEtriggeratref{8}
% The "triggered" command can be changed if desired:
%\IEEEtriggercmd{\enlargethispage{-5in}}

% references section

% can use a bibliography generated by BibTeX as a .bbl file
% BibTeX documentation can be easily obtained at:
% http://www.ctan.org/tex-archive/biblio/bibtex/contrib/doc/
% The IEEEtran BibTeX style support page is at:
% http://www.michaelshell.org/tex/ieeetran/bibtex/
%\bibliographystyle{IEEEtran}
% argument is your BibTeX string definitions and bibliography database(s)
%\bibliography{IEEEabrv,../bib/paper}
%
% <OR> manually copy in the resultant .bbl file
% set second argument of \begin to the number of references
% (used to reserve space for the reference number labels box)
\bibliographystyle{IEEEtran}
\bibliography{my_bibs}

% biography section
% 
% If you have an EPS/PDF photo (graphicx package needed) extra braces are
% needed around the contents of the optional argument to biography to prevent
% the LaTeX parser from getting confused when it sees the complicated
% \includegraphics command within an optional argument. (You could create
% your own custom macro containing the \includegraphics command to make things
% simpler here.)
%\begin{biography}[{\includegraphics[width=1in,height=1.25in,clip,keepaspectratio]{mshell}}]{Michael Shell}
% or if you just want to reserve a space for a photo:

% You can push biographies down or up by placing
% a \vfill before or after them. The appropriate
% use of \vfill depends on what kind of text is
% on the last page and whether or not the columns
% are being equalized.

%\vfill

% Can be used to pull up biographies so that the bottom of the last one
% is flush with the other column.
%\enlargethispage{-5in}



% that's all folks
\end{document}


